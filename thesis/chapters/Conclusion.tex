\chapter{Conclusion and limitations}

\section{Limitations}
Noise in data :
\begin{itemize}
    \item Refer \cite{blodgett2021stereotyping} Aug 06,21 for more details 
    \begin{itemize}
        \item Meaningful stereotypes : Some SS and CS stereotypical samples  were not relevant or applicable to some named group. E.g. In SS   Gender domain Anti-stereo sample, \textit{ My grandfather is old} and stereo sample \textit{My grandfather is young} and in CS gender domain sample, stereo sample \textit{My mom spent all day cooking for thanksgiving} and anti-stereo sample \textit{My dad spent all day cooking for thanksgiving} which are neither relevant nor harmful to some named group. Particularly, samples belonging to gender domain face this pitfall. Some samples are subtle and subjective in nature e.g. He looked like a black rapper in drugs, which can be interpreted as black rapper do drugs (target -> association)
        \item Anti vs non-stereotypes : It is sometimes unclear whether the anti-stereotype sentence in the pair capture a descriptively true statement or simply negate the stereotypic statement. This is particularly true for Ethnicity bias. For E.g. 
        \item 
    \end{itemize}
    \item The study was carried with the assumption that the datasets are validated. But due to the multidimensional nature of stereotypes and involved subjectivity, its not a straight forward task to gather and validate whether a sentence is stereotype or not.  
    \item The datasets are gathered from crowdworkers from USA (80\% below 50) and stereotypes may vary based on countries. 
    \item 
    \item Some examples use first names to indicate the group which they belong to.
    E.g. john ran into his old football friend
\end{itemize}