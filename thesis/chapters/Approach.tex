\chapter{Experiments}

\section{Overview}
\begin{itemize}
    \item Why transfer learning based approach ?
    \item Why classification ?
    \item Final goal of assessing stereotypical social bias in arguments produced by args.me search engine
\end{itemize}

Steps: 
\begin{enumerate}
    \item Collecting stereotypical samples with respect to different domains such as race, gender, profession etc. are gathered from the available sources (CrowS-Pairs \cite{nangia2020crows}, Social bias frames \cite{sap2019social}, StereoSet \cite{nadeem2020stereoset} as seen so far).
    \item Exploratory data analysis for some insights into the different types of biases (gender, race, profession, religion).
    \item Stereotypical dataset target labels interpretation
        \begin{itemize}
            \item Different perspectives of classification labels used 
            \item Anti-stereotypical samples introduced to get the perspective whether a text sequence is socially shared or not socially shared, as stereotypes can be interpreted as pictures in head.
            \item  Including unrelated as a category in classification, why?
        \end{itemize}
    \item Categorization experiments
    \begin{itemize}
        \item Binary classification 
        \item Multi-class classification 
        \item Multi-label classification 
        \item Why going with multi-label ?? When considering anti-stereotypical 
    \end{itemize}
\end{enumerate}
\section{Experimental setup}
\begin{itemize}
    \item Libraries used {Ktrain, huggingface, pytorch lightening}
    \item Classification head and the underlying methodology 
    \item Hyper-parameter search {RayTune}, search space
\end{itemize}
\section{Assessing stereotypical social bias}
\subsection{End-to-end Deep learning approach}
\begin{itemize}
    \item Brief description of language models with main underlying features
    
    Language models :
    \begin{itemize}
        \item GPT-2 \cite{radford2019language}
        
            Features :
    \begin{itemize}
        \item Training data 
        \item Training procedure
        \begin{itemize}
            \item Pre-processing
            \item pre-training
        \end{itemize}
        \item encoded stereotypes (research from stereoset and crows pair)
    \end{itemize}
        \item BERT-base-uncased
        
            Features \cite{devlin2018bert}:
            \begin{itemize}
                \item Training data : BookCorpus (11,038 unpublished books) and English Wikipedia text articles  
                \item Training procedure
                \begin{itemize}
                    \item Pre-processing : Lowercased, tokenized using wordpiece and vocab of size 30,000
                \item pre-training 
                \end{itemize}
                \item encoded stereotypes (research from stereoset and crows pair)
    \end{itemize}
        \item RoBERTa \cite{liu2019roberta}
        
            Features :
            \begin{itemize}
                \item Training data 
                \item Training procedure
                \begin{itemize}
                    \item Pre-processing
                    \item pre-training
                \end{itemize}
                \item encoded stereotypes (research from stereoset and crows pair)
            \end{itemize}
        \item XLnet \cite{yang2019xlnet}
        
            Features :
            \begin{itemize}
                \item Training data 
                \item Training procedure
                \begin{itemize}
                    \item Pre-processing
                    \item pre-training
                \end{itemize}
                \item encoded stereotypes (research from stereo-set and crows pair)
            \end{itemize}
    \end{itemize}
    \item Main features of language models which have impact like underlying training data, architecture and training size as table ??
    \item Why these language model?
\end{itemize}
\subsection{Machine learning approach as baselines}

Questions:

\begin{enumerate}
    \item How does accuracy vary with non-contextual word embeddings (e.g. Glove, flair) to BERT word embedding?
    \item How does accuracy vary with baseline architecture ( applying word embeddings to sentence and taking arithmetic average) to a fine-tuned BERT model ?
\end{enumerate}


    \begin{itemize}
        \item Lexicon - based approach : 
        \begin{itemize}
            \item Project into word embedding space and score each word based on its distance from woman, men ??
            \textbf{Refer }\cite{cryan2020detecting}
        \end{itemize}
        \item Brief description of  models,
        \begin{itemize}
            \item SVM with selected features
            \item Text-CNN with GLove,flair embedding
            \item Random embedding with GRU and LSTM 
        \end{itemize}
        \item Why these model?
    \end{itemize}
\section{Evaluation metrics} \cite{tsoumakas2007multi}
    \begin{itemize}
        \item Evaluation metrics selected for multi-label classification 
        \item Hamming loss 
        \item AUC-ROC score 
        \item Precision, recall, f-measure with respect to different categories
    \end{itemize}
    