\chapter{Data}

In this chapter, we are going to discuss the creation of stereotypical dataset, the categories used with their interpretation and finally an exploratory analysis of the data to gain some insights.

\section{Stereotypical dataset creation and statistics}
As stereotypes are subjective and controversial in nature, collecting annotated data for training machine learning is not such an easy task. Upon some research, we could find three datasets which suits for our goal of assessing stereotypes encoded in text sequences. The three datasets are Stereo Set \cite{nadeem2020stereoset},CrowS-Pairs \cite{nangia2020crows},Social bias frames \cite{sap2019social}. The main dataset used i.e Steroset \cite{nadeem2020stereoset} contain crowdsourced stereotypical samples covering gender, profession, race and religion domains/categories. The dataset was mainly created to test the stereotypes encoded in the pre-trained language models. The main idea is to test the language modeling ability and stereotypical biases encoded using context association test (CAT). In CAT, given a contextual sample containing a target term belonging to one of the domains (e.g. housekeeper(profession)), the likelihood of the language model to choose either a stereotypically biased sample or an unrelated sample is tested. The stereotypically biased sample is further divided into stereotypical and anti-stereotypical. Here, the stereotypical and anti-stereotypical associations  are used to measure the encoded stereotypical biases in language model and unrelated association to measure the language modeling ability. Specifically, tests are carried at the sentence level (intrasentence) and at the discourse level (intersentence). Intrasentence tests are carried with fill-in-the-blank style context sentences, where a context can be filled with either stereotypical, anti-stereotypical or an unrelated attribute instances. At the intersentence level, a contextual sentence with target group can be instantiated either with stereotype, anti-stereotype and unrelated attribute sentences. In all the cases the maximum likelihood of the model to choose a particular attribute (stereo, anti-stereo, unrelated) is used as a measure. Based on this task, the authors of the paper employed crowdworkers (USA-based) via Amazon Mechanical Turk (AMT) to create both intrasentence and intersentence contexts containing target terms from four bias domains, namely (gender, profession, race and religion). 
\begin{itemize}
    \item Why three datasets?? CrowS-Pairs \cite{nangia2020crows}, Social bias frames \cite{sap2019social}, Stereo Set \cite{nadeem2020stereoset}
    \item Stereotypes are considered with respect to historically disadvantaged groups as well as others provided in the list, hence the stereotypic sentences include positive stereotypes as well as negative stereotypes.
\end{itemize}
\subsection{Data preparation}
    \begin{itemize}
        \item Target label definitions used in dataset
        \item Combinations and eliminations done after combining three datasets
    \end{itemize}
\subsection{Exploratory data analysis}